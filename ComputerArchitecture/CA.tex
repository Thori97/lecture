\documentclass[dvipdfmx]{jsarticle}

\usepackage[version=3]{mhchem}
\usepackage{amsmath}
\usepackage[siunitx]{circuitikz}
\usepackage{graphicx}
\usepackage{here}

\setlength\parindent{0pt}

\begin{document}
\begin{enumerate}

\item

略




\item

127+63=189をアセンブルしてみた
\begin{verbatim}
000001_00000_00001_0000000001111111_
000001_00000_00010_0000000000111111_
000000_00010_00001_00011_00000000000_
\end{verbatim}

\item


\begin{enumerate}


\item[3.1]

名称 bgt0\_sub

アセンブリ言語による表現 sub rt, rt, rs + blt r0, rt, dpl

動作 rt$<$-rt-rsをしてrt$>$0ならpc←(pc)+4+dpl

書式 R型

コード 36

\item[3.2]
変更箇所周辺のみ記述する
\begin{verbatim}


  elsif($op eq "ble") { p_b(6, 35); p_r2b($f2, $f3);
   p_b(16, $labels{$f4} - $i - 1); print("\n"); }
 
  elsif($op eq "bgt0_sub") { p_b(6, 36); p_r2b($f2, $f3);
   p_b(16, $labels{$f4} - $i - 1); print("\n"); }
  
  elsif($op eq "j"  ) { p_b(6, 40); p_b(26, $labels{$f2}); print("\n"); }

\end{verbatim}

\item[3.3]
ここも変更箇所周辺のみ記述する
\begin{verbatim}

function	[4:0]	opr_gen;
		input	[5:0]	op;
		input	[4:0]	operation;
		case (op)
      // R型はこっち
			6'd0:	opr_gen = operation;
      // I型でALuを用いるのは以下
			6'd1:	opr_gen = 5'd0;
			6'd4:	opr_gen = 5'd8;
			6'd5:	opr_gen = 5'd9;
			6'd6:	opr_gen = 5'd10;
      // 変更箇所(bgt0_sub)
      6'd36: opr_gen = 5'd2;
			default: opr_gen = 5'h1f;
		endcase
	endfunction
・
・
・
function	[31:0]	calc;
		input	[5:0]	op;
		input	[31:0]	alu_result, dpl_imm, dm_r_data, pc;
		case (op)
      // 変更箇所(bgt0_sub)
			6'd0, 6'd1, 6'd4, 6'd5, 6'd6, 6'd36:	calc = alu_result;
			6'd3:	calc = dpl_imm << 16;
・
・
・

function	[31:0]	npc;
		input	[5:0]	op;
		input	[31:0]	reg1, reg2, branch, nonbranch, addr;
		case (op)
			6'd32:	npc = (reg1 == reg2) ? branch : nonbranch;
			6'd33:	npc = (reg1 != reg2) ? branch : nonbranch;
			6'd34:	npc = (reg1 < reg2)  ? branch : nonbranch; 
			6'd35:	npc = (reg1 <= reg2) ? branch : nonbranch;
      // 変更箇所(bgt0_sub)
      6'd36:  npc = (reg1 > reg2)  ? branch : nonbranch;
			6'd40, 6'd41:	npc = addr;
・
・
・
function	[4:0]	wreg;
		input	[5:0]	op;
		input	[4:0]	rs, rt, rd;
		case (op)
			6'd0:	wreg = rd;
			6'd1, 6'd3, 6'd4, 6'd5, 6'd6, 6'd16, 6'd18, 6'd20:	wreg = rt;
      // 変更箇所(bgt0_sub)
      6'd36: wreg = rs;

\end{verbatim}

\item[3.4]
\begin{verbatim}
addi r1, r0, 10
addi r2, r0, 1
addi r3, r0, 0
loop_start : add r3, r2, r3
addi r2, r2, 1
ble r2, r1, loop_start
j, loop_end
loop_end : addi, r3, r3, 0
\end{verbatim}
を翻訳して
\begin{verbatim}
00001_00000_00001_0000000000001010_
000001_00000_00010_0000000000000001_
000001_00000_00011_0000000000000000_
000000_00010_00011_00011_00000000000_
000001_00010_00010_0000000000000001_
100011_00010_00001_1111111111111101_
101000_00000000000000000000000111_
000001_00011_00011_0000000000000000_
\end{verbatim}

\item[3.5]
\begin{verbatim}
addi r1, r0, 10
addi r2, r0, 1
addi r3, r0, 0
loop_start : add r3, r1, r3
bgt0_sub,r1,r2, loop_start
j, loop_end
loop_end : addi, r3, r3, 0
\end{verbatim}
を翻訳して
\begin{verbatim}
000001_00000_00001_0000000000001010_
000001_00000_00010_0000000000000001_
000001_00000_00011_0000000000000000_
000000_00001_00011_00011_00000000000_
100100_00001_00010_1111111111111110_
101000_00000000000000000000000110_
000001_00011_00011_0000000000000000_
\end{verbatim}

\item[3.6]

bgt0\_subを導入したことで1ループで1クロック性能向上するので、性能向上はN


\end{enumerate}



\end{enumerate}

\end{document}