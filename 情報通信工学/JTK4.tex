\documentclass[dvipdfmx]{jsarticle}

\usepackage[version=3]{mhchem}
\usepackage{amsmath}
\usepackage[siunitx]{circuitikz}
\usepackage{graphicx}
\usepackage{here}

\setlength\parindent{0pt}

\begin{document}
\title{情報通信工学レポート}
\author{工学部電子情報工学科3年 03190449  堀 紡希}
\date{\ 10月23日}
\maketitle

\section*{レポート課題1}

\begin{enumerate}
\item[(a)]
\[f_{FM}(t) = A_{c}\cos{\left(\omega_{c}t+k_{f}\int _{-\infty} ^{t} m_{1}(\tau)d\tau \right)}\]
が、\[f_{PM}(t) = A_{c}\cos{\omega_{c} t+k_{p}m_{2}t}\]
と等しいので、
\[k_{f}\int _{-\infty} ^{t} m_{1}(\tau)d\tau = k_{p}m_{2}(t)\]
両辺tで微分して、
\[k_{f}m_{1}(t) = k_{p}\dot{m_{2}}(t)\]
が$m_{1}$と$m_{2}$の満たす条件である。
\item[(b)]


\end{enumerate}
\section*{レポート課題2}
\begin{enumerate}
\item[(a)]
\item[(b)]
\item[(c)]
\end{enumerate}



\section*{レポート課題3}
\begin{enumerate}
\item[(a)]

\item[(b)]
\end{enumerate}
\end{document}










