\documentclass[dvipdfmx]{jsarticle}

\usepackage[version=3]{mhchem}
\usepackage{amsmath}
\usepackage[siunitx]{circuitikz}
\usepackage{graphicx}
\usepackage{here}

\setlength\parindent{0pt}

\begin{document}
\title{統計的機械学習レポート}
\author{工学部電子情報工学科3年 03190449  堀 紡希}
\date{\ 6月11日}
\maketitle

\section{ID:01}

\begin{verbatim}
import math

def beta(a, b, p):
    return (math.gamma(a+b)*p**(a-1)*(1-p)**(b-1))/(math.gamma(a)*math.gamma(b))

from scipy import integrate

def beta52(p):
    return beta(5, 2, p)
print(1-integrate.quad(beta52, 0, 0.5)[0])

print(1-integrate.quad(beta52, 0, 0.8)[0])

def beta0101(p):
    return beta(4.1, 1.1, p)
print(1-integrate.quad(beta0101, 0, 0.8)[0])

def beta96(p):
    return beta(9, 6, p)
print(1-integrate.quad(beta96, 0, 0.8)[0])
\end{verbatim}


を実行して

$p(\pi>0.5 | data) \approx  0.89$

$p(\pi>0.8 | data) \approx  0.34$

beta(0.1, 0.1)の時

$p(\pi>0.8 | data) \approx  0.56$

beta(5, 5)の時

$p(\pi>0.8 | data) \approx  0.043$であった。

\section{ID:02}


$\pi$の事後分布が$Beta(\pi|16, 6)$

負の二項分布の期待値$E_{NB(x|\pi)}[x] = k\frac{1-\pi}{\pi}$

より2人陽性が出るまでの陰性の人の数の期待値が$E_{NB(x|\pi)}[x] = 2\frac{1-\pi}{\pi}$


\[E[x|data] = \int_{0}^{1}2\frac{1-\pi}{\pi}Beta(\pi|16, 6)d\pi\]

$\pi$が0.1(90\%で二人以上当たる)を仮定すると

\[ \int^{0.1}_{0} 18 Beta(\pi | 16, 6)d\pi \approx 0.014\]


よって18人選んでくると約98.5\%以上の確率で陽性の人が二人以上いる。

\begin{verbatim}
def beta616(p):
    return beta(6, 16, p)
print(integrate.quad(beta616, 0, 0.1)[0])
\end{verbatim}

ID:01に追加してこれで実行した。
\section{ID:03}


よくわからなかったので後でやります。











\end{document}