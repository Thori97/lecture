\documentclass[dvipdfmx]{jsarticle}

\usepackage[version=3]{mhchem}
\usepackage{amsmath}
\usepackage[siunitx]{circuitikz}
\usepackage{graphicx}
\usepackage{here}
\usepackage{mathrsfs}

\setlength\parindent{0pt}

\begin{document}
\title{統計的機械学習レポート}
\author{工学部電子情報工学科3年 03190449  堀 紡希}
\date{\ 7月9日}
\maketitle

\section{レポート課題1}


仮定より
\[f(x_{\ast}) \sim \mathscr{N}(m(x_{\ast}),\kappa(x_{\ast},x_{\ast})) \]

\[f(x_{\ast})|f(x_{1}),\dots,f(x_{n})\sim \mathscr{N}(\mu_{n}(x_{\ast}),\sigma_{n}^{2}(x_{\ast})) \]
である。

ガウス分布の条件付き分布の$\hat{\mu},\hat{\Sigma}$の関係式

\[\hat{\mu_{2}}=\mu_{2}+\Sigma_{21}\Sigma_{11}^{-1}(x_{1}-\mu_{1})\]
\[\hat{\Sigma_{2}}=\Sigma_{22}-\Sigma_{21}\Sigma_{11}^{-1}\Sigma_{12}\]

より

\[\mu_{n}(x_{\ast}) = m(x_{\ast})+\kappa(x_{\ast})^{T}K(x_{1:n})^{-1}(y_{1:n}-m(x_{1:n}))\]
\[\sigma_{n}^{2}(x_{\ast})=\kappa(x_{\ast},x_{\ast})-\kappa(x_{\ast})^{T}K(x_{1:n})^{-1}\kappa(x_{\ast})\]
ただし
\[\kappa(x_{\ast})=(\kappa(x_{\ast},x_{1}),\kappa(x_{\ast},x_{2}),\dots,\kappa(x_{\ast},x_{n}))^{T}\]









\end{document}